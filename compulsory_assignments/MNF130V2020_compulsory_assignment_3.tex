\documentclass[12pt]{article}

\usepackage[T1]{fontenc}
\usepackage{a4wide}
\usepackage{mathptmx}
\usepackage{graphicx}
\usepackage[hidelinks]{hyperref}
\usepackage{mathbbol,bm,amsmath,amsthm,amsfonts,amssymb}
\usepackage{mathtools}
\usepackage{booktabs}
\usepackage{fancybox}
\usepackage{tikz}
\usepackage[shortlabels]{enumitem}
\usepackage{fancyhdr}

\setlist[itemize]{label={$\bullet$}, topsep=-0.2ex,
  partopsep=0pt, parsep=4pt, itemsep=0pt, leftmargin=4.5mm,
  labelsep=0.5ex}

\setlist[enumerate]{align=left, listparindent=0pt}

\newcommand{\R}{\mathbb{R}}
\newcommand{\N}{\mathbb{N}}
\newcommand{\Z}{\mathbb{Z}}
\renewcommand{\iff}{\;\leftrightarrow\;}
\newcommand{\floor}[1][x]{\lfloor #1\rfloor}
\newcommand{\ceil}[1][x]{\lceil #1\rceil}
\newcommand{\mat}[1][A]{\text{\textbf{#1}}}
\newcommand{\I}{\mathbb{1}}

\pagestyle{fancy}
\fancyhf{}
\rhead{Mandatory assignment 3}
\lhead{MNF130}
\chead{\today}


\makeatletter
\newcommand*{\bdiv}{%
  \nonscript\mskip-\medmuskip\mkern5mu%
  \mathbin{\operator@font div}\penalty900\mkern5mu%
  \nonscript\mskip-\medmuskip
}
\makeatother

\renewcommand{\thesubsection}{\arabic{subsection}}
\renewcommand{\thesubsubsection}{\alph{subsubsection}}

\setlength{\parindent}{0pt}
\setlength{\parskip}{1ex plus 0.5ex minus 0.2ex}

\newenvironment{answer}{\color{blue}}{}
\newcommand{\bl}[1][T]{{\color{blue}#1}}

\graphicspath{{./}{../images/}}

\title{\huge Compulsory Assignment 3}

\author{\LARGE MNF130V2020\\}


\date{\Large\textbf{Due on: Friday 13 March 2020, 14:00}}


\begin{document}

\maketitle

\bigskip

  \vspace*{15mm}

  {\large
  \begin{minipage}{\linewidth}
    \begin{itemize}
    \item Submit your assignment by leaving it in the box marked
      \textbf{MNF130} at the \textbf{reception of the Department of
        Informatics (Datablokken, 4th floor)}. The box will be
      available from Monday 9 March.  \smallskip
    \item If you cannot deliver the assignment in person, only use a
      \textbf{UiB Pullprint scanner} and \textbf{submit on
        Mitt}. Assignments sent by email, or from camera pictures
      instead of scans, or otherwise illegible, will not count as a
      valid submission and will not be graded.
      \smallskip
    \item Write your answers \textbf{one-sided} (don't use both sides
      of a page), and start a new page for every exercise.
      \smallskip
    \item Write \textbf{your name} on every page.
      \smallskip
    \item You may write your answers in English or Norwegian.
      \smallskip
    \item The assignment covers the entire syllabus covered during the
      lectures so far.
      \smallskip
    \item The assignment is scored on \textbf{30 points}. Hence you
      need to score \textbf{at least 10.5 points to pass}.
  \end{itemize}
  \end{minipage}
  }

\bigskip

\bigskip

\newpage

\subsection{Propositional and predicate logic (8 points)}

\begin{enumerate}[a)]
\item Fill out the truth table:\\
 \begin{tabular}[t]{|c|c|c|c|c|c|c|c|}
   \toprule
   $p$ & $q$ & $r$ & $p\to q$ & $q\to r$ & $(p\to q)\land (q\to r)$ 
   & $p\to r$   & $[(p\to q)\land (q\to r)]\to (p\to r)$\\
   \midrule 
   T & T & T & T & T & T & T & T \\
   \hline
   T & T & F & T & F & F & F & T \\
   \hline
   T & F & T & F & T & F & T & T\\
   \hline 
   T & F & F & F & T & F & F & T\\
   \hline 
   F & T & T & T & T & T & T & T\\
   \hline 
   F & T & F & T & F & F & T & T\\
   \hline 
   F & F & T & T & T & T & T & T\\
   \hline 
   F & F & F & T & T & T & T & T\\
   \bottomrule
 \end{tabular}
%\end{center}
\item Let $p,q,r$ be propositions. Is the compound proposition
  $[(p\to q)\land (q\to r)]\to (p\to r)$ a tautology?
  Explain why/why not.

  \textbf{The compound proposition is a tautology, because it is true for all cases in the truth table.\\
    By definition, a tautology is an assertion that is true in every possible case/interpretation}.
  \qed
\medskip

\item Let $p,q,r$ be propositions. Use basic logical equivalences to
  prove that the compound propositions $(p\land \lnot q)\to  r$
  and $p\to (q\lor r)$ are logically equivalent.

\textit{Write as disjunction}. \\
$(p \wedge \neg q) \rightarrow r \equiv \neg (p \wedge \neg q) \vee r$\\
\textit{De Morgan}. \\
$ \neg (p \wedge \neg q) \vee r \equiv (\neg p \vee q) \vee r$\\
\textit{Assiociative laws}. \\
$ (\neg p \vee q) \vee r \equiv \neg p \vee ( q \vee r)$\\
\textit{Logical equivalence for conditional statements}. \\
$ \neg p \vee (p \vee r) \equiv p \rightarrow (q \vee r) $\\
\qed


\medskip

\item What are the truth values of the statements
  $\forall n\exists m(n+m=0)$ and $\exists n\forall m(n<m^2)$ if the
  domain for all variables consists of $\Z$, the set of all integers?
  Explain your answer.

  \begin{itemize}
  \item The first statement, $\forall n\exists m(n+m=0)$ is \textbf{true}, because the domain of integers contains a negative/positive counterpart for every value. For the particular statement, it means that for all $n$, there exists a counterpart such that the sum will be 0.
    This works because there only has to exist a single $m$ for any $n$ value.
  \item The second statement, $\exists n\forall m(n<m^2)$ is \textbf{true}, because there has to exist a value of $\Z$ such that all values in $\Z$ squared are greater that this number. You can pick $n$ to be a negative number, and all $m$ values will be strictly greater. So the statement is true.  
  \end{itemize}
  \qed
\end{enumerate}

\bigskip

\subsection{Set theory and functions (8 points)}
\label{sec:set-theory-functions}

\def\firstcircle{(0,0) circle (1.5cm)}
\def\secondcircle{(45:2cm) circle (1.5cm)}
\def\thirdcircle{(0:2cm) circle (1.5cm)}

\colorlet{circle edge}{blue!50}
\colorlet{circle area}{blue!20}
\tikzset{filled/.style={fill=circle area, draw=circle edge, thick}, outline/.style={draw=circle edge, thick}}


\begin{enumerate}[a)]
\item Let $A, B, C$ be sets. Draw a Venn diagram and color the region
  $(A - C) \cap (C- B)$.  Prove (using set identities) or disprove
  (give a counterexample) that $(A - C) \cap (C- B) = \emptyset$.

  \begin{tikzpicture}
    \draw \firstcircle node[below] {$A$};
    \draw \secondcircle node[above] {$B$};
    \draw \thirdcircle node[below] {$C$};
  \end{tikzpicture}

  Let $x \in (A - C) \wedge x \in (C - B)$. \\
  For the first part of the intersection, $x$ is in $A$ and not in $C$, for the second part of the intersection, $x$ is in $C$ and not in $B$. \\

  Expanding this, we get: \\
  $x \in C \wedge \neg (x \in C)$

  Which is \textbf{F} by the negation law. 

  The proposition now looks like: \\
  $x \in A \wedge \neg (x \in B) \wedge F$. \\

  With set identities, this can be written as: \\
  $(A - C) \cap (C - B) \equiv (A \cap \overline{C}) \cap (C \cap \overline{B}) \equiv (A \cap \overline{B}) \cap (C \cap \overline{C})$ 

  $ (A \cap \overline{B}) \cap (C \cap \overline{C}) \equiv (A\cap \overline{B}) \cap \emptyset \equiv \emptyset$

  This means that $(A-C) \cap (C-B) \subseteq \emptyset$.
  And thus the initial proposition is a subset of the empty set.
  \qed
  
  % \begin{itemize}
  % \item $
  % \end{itemize}
  
\medskip

\item Let $A, B, C$ be sets. Draw a Venn diagram and color the region
  $(A - C) \cap (B- C)$. Prove (using set identities) or disprove
  (give a counterexample) that $(A - C) \cap (B- C) = \emptyset$.

  \begin{tikzpicture}
    \draw \firstcircle node[below] {$A$};
    \draw \secondcircle node[above] {$B$};
    \draw \thirdcircle node[below] {$C$};

    \begin{scope}
      \clip \secondcircle;
      \clip \firstcircle;
      \draw[filled, even odd rule] \firstcircle \thirdcircle;
    \end{scope}
  \end{tikzpicture}

  Looking at the diagram, I will try to create a counterexample, because it looks like that's whats needed... \\

  Let $A = B = \{1\}$ and $ C = \emptyset$.

  $A - C = A - \emptyset = A = B = \{1\}$. \\
  $B - C = B - \emptyset = B = A = \{1\}$. \\

  $(A - C) \cap (B - C) = A \cap B = A \cap A = A = B = \{1\}$.

  $\{1\} \neq \emptyset$ and therefore we disproved the proposition by counterexample.
  \qed
  
\medskip

\item Let $f(x)=x^2$ be a function from the set of real numbers to the
  set of real numbers. Is $f$ one-to-one (injective)? Onto
  (surjective)? A one-to-one correspondence (bijective)? Explain
  why/why not.
\end{enumerate}

\begin{itemize}
\item It is \textbf{not} injective for $\R$ because multiple inputs will map to the same output. A quick example would be 3 and -3.
\item It is also \textbf{not} surjective because the negative co-domain $\R^-$ does not have pre-images mapping to them. No input $x$ will map to a value $y < 0$. Thus is not surjective.
\item It is also \textbf{not} bijective because it is neither injective or surjective. \qed
\end{itemize}

\newpage

\subsection{Number theory (6 points)}
\label{sec:number-theory}

\begin{enumerate}[a)]
\item Let $a$ be an integer that is not divisible by 3. Prove that
  $(a+1)(a+2)$ is divisible by 3.

\medskip

\item Use the Euclidean algorithm to find $\gcd(252,356)$. \\

  $356/252 = 1$, the remainder is $104$. $356$ can be written as $252 \cdot 1 + 104$. Continue with gcd(252,104). \\

  $252/104 = 2$, remainder $44$. $252 = 104 * 2 + 44$. gcd(104,44). \\

  $104/44 = 2$, remainder $16$. $104 = 44 * 2 + 16$. gcd(44, 16). \\

  $44/16 = 2$, remainder $12$. $44 = 16 * 2 + 12$. gcd(16,12) \\

  $ 16 / 12 = 1$, remainder $4$. $16 = 12 * 1 + 4$. gcd(12,4) \\

  $12/4 = 3$. Greater common denominator with 356 and 252 is \textbf{3}.
  \qed
  
\medskip

\item Find each of these values:
  \begin{itemize}
  \item $(177\bmod 31 + 270\bmod 31)\bmod 31$ \\

    177 mod 31 = 22 and 270 mod 31 = 22.

    Simplify the equation as $(22 + 22)$ mod $31$ = 44 mod 31 = \textbf{13}.
    \qed
    \medskip
  \item $\bigl[5(99^2\bmod 32)\bigr]\bmod 15$

    $99^2$ = 9801. \\

    9801 mod 32 = 9. \\

    9 * 5 = 45. \\

    45 mod 15 = \textbf{0}.
    \qed
    
  \end{itemize}
\end{enumerate}

%\newpage
\bigskip

\subsection{Induction (8 points)}
\label{sec:induction}

Use mathematical induction to prove that
  $$\sum_{k=1}^n k^2 = \frac{n(n+1)(2n+1)}{6}$$
for all integers $n\geq 1$.   \\

Let $P(n)$ =   $$\sum_{k=1}^n k^2 = \frac{n(n+1)(2n+1)}{6}$$ 


\begin{itemize}
\item Basis step, verify $P(1)$ (smallest $n$ in this case is 1).
  $$
  P(1) = \sum_{k=1}^1 k^2 = \frac{1(1+1)(2\cdot 1+1)}{6} \Rightarrow 1^2 = \frac{6}{6} = 1
  $$
\item Next, we assume P(m) is true for an arbitrary $m \geq 1$. \\
  To write the function in terms of m: \\

  $$P(m) = \sum_{k=1}^m k^2 = \frac{m(m+1)(2m+1)}{6}$$

\item $P(m+1)$ can be written as: \\

  $$
  P(m+1) = \sum_{k=1}^m k^2 + (m+1)^2 = \frac{m(m+1)(2m+1)}{6} + (m+1)^2
  $$

  $$
  = \frac{m(m+1)(2m+1)+6(m+1)^2}{6} = \frac{m(m+1)(2m+1)+6(m+1)(m+1)}{6}
  $$

  $$
  = \frac{(m+1)(m(2m+1)+6(m+1))}{6} = \frac{(m+1)(m+2)(2(m+1)+1)}{6} 
  $$

  $$
  = \frac{(m+1)((m+1)+1)(2(m+1)+1)}{6}
  $$

  Now we have shown that the basis is verified, and it will hold for $P(m)$ and some $P(m+1)$ where $m \geq 1$.\\
  \qed
  
\end{itemize}


\end{document}
%%% Local Variables:
%%% mode: latex
%%% TeX-master: t
%%% End:
