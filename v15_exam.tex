\documentclass[12pt]{article}

\usepackage[T1]{fontenc}
\usepackage{mathptmx}
\usepackage{graphicx}
\usepackage[hidelinks]{hyperref}
\usepackage{mathbbol,bm,amsmath,amsthm,amsfonts,amssymb}
\usepackage{mathtools}
\usepackage{booktabs}
\usepackage{fancyhdr}
\usepackage[mathscr]{euscript}
\usepackage[margin=0.5in]{geometry}
\usepackage{tikz}

\graphicspath{ {./images/}}

% Used for fancy bullet lists. Not really using it now
% \usepackage[shortlabels]{enumitem}
% \setlist[itemize]{label={$\bullet$}, topset=0.2ex, partopset=0pt, parsep=4pt, itemsep=0pt, leftmargin=4.5mm, labelsep=0.5ex}
% \setlist[enumerate]{align=left, listparindent=0pt}

\newcommand{\PP}{\mathscr{P}}
\newcommand{\R}{\mathbb{R}}
\newcommand{\N}{\mathbb{N}}
\newcommand{\Z}{\mathbb{Z}}
\renewcommand{\iff}{\;\leftrightarrow\;}
\newcommand{\floor}[1][x]{\lfloor #1\rfloor}
\newcommand{\ceil}[1][x]{\lceil #1\rceil}
\newcommand{\mat}[1][A]{\text{\textbf{#1}}}
\newcommand{\I}{\mathbb{1}}

\renewcommand{\thesubsection}{\arabic{subsection}}
\renewcommand{\thesubsubsection}{\alph{subsubsection}}

\title{\huge V15 exam notes}

\author{\LARGE MNF130V2020}

\begin{document}
\subsection{Opgave 1}

\begin{itemize}
\item True, pick $m = n^2+1$
\item True, pick $n = 0$
\item True, pick $x = 3, y = 1$. 
\item False
\item True
\item True
\item False
\end{itemize}

\subsection{Opgave 2}
\begin{itemize}
\item 1,2,5,6
\item 2,4,5,6,7
\item 1,3,4,6,7,8
\item 3,4,5,6,7
\item 2,3,4,5,7,8
\item 1,4
\item 2,3,4,5,6,7,8
\item 1,2,3,4,5,6
\end{itemize}

\subsection{Opgave 3}

\textbf{b)} \\
$$
(p \rightarrow q) \vee (\neg p \rightarrow q) \equiv (\neg p \vee q) \vee (p \vee q)
\equiv (\neg p \vee p) \vee (q \vee q) \equiv \textbf{T} \vee q \equiv \textbf{T}
$$

\subsection{Opgave 4}

Use floor division and modulo operator \\
\begin{itemize}
\item 1 | 66 | 1 | 66
\item -2 | 21 | -2 | 21
\item  -5 | 2 | -5 | 2
\item 11 | 8 | 11 | 8
\end{itemize}

\newpage

\subsection{Opgave 5} 

\textbf{a)} \\
$$
a_2 = 6 \cdot 3 - 9 \cdot 1 = 9
$$
$$
a_3 = 6 \cdot 9 - 9 \cdot 3 = 54 - 27 = 27
$$

$$
a_4 = 6 \cdot 27 - 9 \cdot 9 = 81
$$

\textbf{b)} \\

$$
n = 0, a_0 = 3^0 = 1
$$
$$
n = 1, a_1 = 3^1 = 3
$$
So the basis steps are correct. \\

\textbf{c)} \\

We now assume that $a_j = 3^j$ for any $j \leq k$ 

\textbf{d)} \\

Now we have to show that $a_{k+1} = 3^{k+1}$. \\

By using the formula, we can write: \\

$$
a_{k+1} = 6a_k - 9 \cdot a_{k-1}
$$

$$
= 6 \cdot 3^k - 9 \cdot 3^{k-1} = 2 \cdot 3 \cdot 3^k - 3^2 \cdot 3^{k-1} = 3^{k+1} (2-1)
= 3^{k+1}
$$

\subsection{Opgave 6}

Let $A = \{1,2\}$, $B = \{2,3\}$. \\
$$
A \cap B = \{2\}
$$
$$
A + B = \{1,2,2,3\}
$$
$$
A + B - (A \cap B) = \{1,2,2,3\} - \{2\} = \{1,2,3\}
$$

\end{document}